\chapter{Conexión al broker MQTT con FastAPI y AWS IoT SDK para Python}
\label{AppendixE}

La conexión al broker MQTT se realiza para permitir la comunicación entre el
servidor IoT y el broker MQTT.

El código \ref{cod:aws_iot_policy} muestra una política de acceso en AWS IoT
Core que habilita al cliente a conectarse, publicar, suscribirse y recibir
mensajes en cualquier tópico.

\begin{lstlisting}[label=cod:aws_iot_policy,caption=Ejemplo de política de acceso en AWS IoT Core., language=JSON]
{
    "Version": "2012-10-17",
    "Statement": [
        {
            "Effect": "Allow",
            "Action": [
                "iot:Connect",
                "iot:Publish",
                "iot:Subscribe",
                "iot:Receive"
            ],
            "Resource": 
                "arn:aws:iot:*:*:*"
        }
    ]
}
\end{lstlisting}

El código \ref{cod:mqtt_connection} muestra el proceso de conexión a AWS IoT
Core y la implementación de la lógica de publicación y suscripción a los
tópicos.

En este código, se definen los métodos para conectar al broker, publicar y
suscribirse a tópicos, y manejar los mensajes recibidos. Se implementó un
cliente MQTT que interactúa con AWS IoT Core, gestionando la comunicación con
los nodos sensores y actuadores. Además, se incorporaron métodos para recibir
datos de estos dispositivos, enviarles comandos y almacenar la información en
la base de datos MongoDB.

%%%%%%%%%%%%%%%%%%%%%%%%%%%%%%%%%%%%%%%%%%%%%%%%%%%%%%%%%%%%%%%%%%%%%%%%%%%%%
% parámetros para configurar el formato del código en los entornos lstlisting
%%%%%%%%%%%%%%%%%%%%%%%%%%%%%%%%%%%%%%%%%%%%%%%%%%%%%%%%%%%%%%%%%%%%%%%%%%%%%
\lstset{ %
    backgroundcolor=\color{white},   % choose the background color; you must add \usepackage{color} or \usepackage{xcolor}
    basicstyle=\footnotesize,        % the size of the fonts that are used for the code
    breakatwhitespace=false,         % sets if automatic breaks should only happen at whitespace
    breaklines=true,                 % sets automatic line breaking
    captionpos=b,                    % sets the caption-position to bottom
    commentstyle=\color{mygreen},    % comment style
    deletekeywords={...},            % if you want to delete keywords from the given language
    %escapeinside={\%*}{*)},          % if you want to add LaTeX within your code
    %extendedchars=true,              % lets you use non-ASCII characters; for 8-bits encodings only, does not work with UTF-8
    %frame=single,	                % adds a frame around the code
    keepspaces=true, keywordstyle=\color{blue}, language=[ANSI]C, % keeps spaces in text, useful for keeping indentation of code (possibly needs columns=flexible)% keyword style% the language of the code
    %otherkeywords={*,...},           % if you want to add more keywords to the set
    numbers=left, numbersep=5pt, numberstyle=\tiny\color{mygray},
    rulecolor=\color{black}, showspaces=false, showstringspaces=false,
    showtabs=false, stepnumber=1, stringstyle=\color{mymauve}, tabsize=2,
    title=\lstname, morecomment=[s]{/*}{*/} }% where to put the line-numbers; possible values are (none, left, right)% how far the line-numbers are from the code% the style that is used for the line-numbers% if not set, the frame-color may be changed on line-breaks within not-black text (e.g. comments (green here))% show spaces everywhere adding particular underscores; it overrides 'showstringspaces'% underline spaces within strings only% show tabs within strings adding particular underscores% the step between two line-numbers. If it's 1, each line will be numbered% string literal style% sets default tabsize to 2 spaces% show the filename of files included with \lstinputlisting; also try caption instead of title

\lstdefinelanguage{PythonUTF8}[]{Python}{
literate={á}{{\'a}}1 {é}{{\'e}}1 {í}{{\'i}}1 {ó}{{\'o}}1 {ú}{{\'u}}1
{Á}{{\'A}}1 {É}{{\'E}}1 {Í}{{\'I}}1 {Ó}{{\'O}}1 {Ú}{{\'U}}1
{ñ}{{\~n}}1 {Ñ}{{\~N}}1
}

\definecolor{mygreen}{rgb}{0,0.6,0}
\definecolor{mygray}{rgb}{0.5,0.5,0.5}
\definecolor{mymauve}{rgb}{0.58,0,0.82}

\begin{lstlisting}[label=cod:mqtt_connection,caption=Definición de Clase para cliente MQTT., language=PythonUTF8]
    import asyncio
    import json
    import os
    # Importamos el cliente MQTT de AWS
    from AWSIoTPythonSDK.MQTTLib import AWSIoTMQTTClient
    # Importamos los modelos de datos
    from models.actuator import Actuator
    from models.actuator_data import ActuatorData
    from models.sensor_consumption import ConsumptionSensor
    from models.sensor_consumption_data import ConsumptionSensorData
    from models.sensor_environmental import EnvironmentalSensor
    from models.sensor_environmental_data import EnvironmentalSensorData
    from models.sensor_nutrient_solution import NutrientSolutionSensor
    from models.sensor_nutrient_solution_data import NutrientSolutionSensorData
    # Importamos la configuración de AWS
    from mqtt.aws_config import AWS_ENDPOINT, MQTT_CLIENT_ID
    # Importamos el cliente WebSocket
    from websocket_manager import websocket_manager
    import datetime

    TOPIC_PUB_MODEL_MAP = {
        "environmental/sensor/pub": EnvironmentalSensorData,
        "nutrient/solution/sensor/pub": NutrientSolutionSensorData,
        "consumption/sensor/pub": ConsumptionSensorData,
        "actuators/pub": ActuatorData,
    }

    TOPIC_SUB_MODEL_MAP = {
        "environmental/sensor/sub": EnvironmentalSensor,
        "nutrient/solution/sensor/sub": NutrientSolutionSensor,
        "consumption/sensor/sub": ConsumptionSensor,
        "actuators/sub": Actuator,
    }

    # Método asíncrono para insertar datos en la base de datos
    async def insert_sensor_data(payload, model_class):
        try:
            sensor_data = model_class(**payload)
            await sensor_data.insert()
            print("Datos guardados en la base de datos.")
        except Exception as e:
            print(f"Error al guardar datos en la base de datos: {e}")

    # Método asíncrono para actualizar el intervalo de reporte
    async def update_device_interval(device_code, new_interval, model_class, device_type="sensor"):
        try:
            print(f"Actualizando {device_type} {device_code} a {new_interval}s en BD...")
            
            # Convertir a entero explícitamente
            interval_value = int(new_interval)
            # Buscar el dispositivo usando Beanie
            query_field = "sensor_code" if device_type == "sensor" else "actuator_code"
            device = await model_class.find_one({query_field: device_code})
            
            if device:
                # Actualizar usando el método set de Beanie
                await device.set({"seconds_to_report": interval_value})       
                print(f"Intervalo actualizado para {device_type} {device_code}")
                return True
            else:
                print(f"{device_type.capitalize()} {device_code} no encontrado en BD")
                return False
                
        except ValueError as e:
            print(f"Error de validación: {e}")
        except Exception as e:
            print(f"Error actualizando BD: {str(e)}")
        return False
            
    # Método para procesar mensajes entrantes de sensores y actuadores
    async def process_sensor_message_pub(topic, payload):
        try:
            print(f"Mensaje recibido desde {topic}")
            print(f"Data: {payload}")

            model_class = TOPIC_PUB_MODEL_MAP.get(topic)
            if model_class:
                print(f"Procesando mensaje de {topic}")
                await insert_sensor_data(payload, model_class)
                
                # Determinar el tipo de sensor para WebSocket
                sensor_type = None
                if "environmental/sensor/pub" in topic:
                    sensor_type = "environmental"
                elif "nutrient/solution/sensor/pub" in topic:
                    sensor_type = "nutrient_solution"
                elif "consumption/sensor/pub" in topic:
                    sensor_type = "consumption"
                elif "actuators/pub" in topic:
                    sensor_type = "actuators"
                
                # Verificar si el tipo de sensor es válido
                if sensor_type:
                    # Actualizar cache y enviar a WebSocket
                    websocket_manager.update_cache(sensor_type, payload)
                    await websocket_manager.broadcast({
                        "type": sensor_type,
                        "data": payload,
                        "timestamp": datetime.datetime.now().isoformat()
                    })
            else:
                print(f"Tópico no reconocido: {topic}")
        except Exception as e:
            print(f"Error inesperado: {e}")

    # Método para procesar mensajes de control de sensores y actuadores
    async def process_sensor_message_sub(topic, payload):
        try:
            print(f"Mensaje recibido desde {topic}")
            print(f"Data: {payload}")

            model_class = TOPIC_SUB_MODEL_MAP.get(topic)
            if not model_class:
                print(f"Tópico no reconocido: {topic}")
                return

            # Determinar si es un sensor o actuador
            is_actuator = "actuators/sub" in topic
            device_type = "actuator" if is_actuator else "sensor"
            code_field = "actuator_code" if is_actuator else "sensor_code"
            
            # Verificar que el campo de código exista en el payload
            if code_field not in payload:
                print(f"Falta campo {code_field} en el mensaje de {device_type}")
                return

            # Solo procesamos si es una confirmación del dispositivo
            if payload.get("interval") == "OK" and payload.get("seconds_to_report"):
                print(f"Confirmación válida recibida del {device_type}")
                await update_device_interval(
                    device_code=payload[code_field],
                    new_interval=payload["seconds_to_report"],
                    model_class=model_class,
                    device_type=device_type
                )
            else:
                print(f"Mensaje recibido (esperando confirmación del {device_type})")
        except Exception as e:
            print(f"Error procesando mensaje: {e}")
                
    # Clase para manejar la conexión y comunicación con AWS IoT Core
    class AWSMQTTClient:
        def __init__(self):
            CERTS_DIR = os.getenv("CERTS_DIR", "certificates")
            self.ROOT_CRT = os.path.join(CERTS_DIR, "root.crt")
            self.CLIENT_CRT = os.path.join(CERTS_DIR, "client.crt")
            self.CLIENT_KEY = os.path.join(CERTS_DIR, "client.key")

            # Verificar existencia de certificados
            for cert in [self.ROOT_CRT, self.CLIENT_CRT, self.CLIENT_KEY]:
                if not os.path.exists(cert):
                    raise FileNotFoundError(f"El archivo {cert} no fue encontrado.")

            # Configurar cliente MQTT
            self.client = AWSIoTMQTTClient(MQTT_CLIENT_ID)
            self.client.configureEndpoint(AWS_ENDPOINT, 8883)
            self.client.configureCredentials(self.ROOT_CRT, self.CLIENT_KEY, self.CLIENT_CRT)
            self.client.configureConnectDisconnectTimeout(10)
            self.loop = asyncio.get_event_loop()

        def connect(self):
            print("Estableciendo conexión con AWS IoT Core.")
            self.client.connect()
            print("Cliente MQTT Conectado.")

        def disconnect(self):
            self.client.disconnect()
            print("Cliente Desconectado")

        def publish(self, topic, message):
            payload = json.dumps(message) if isinstance(message, dict) else str(message)
            self.client.publish(topic, payload, 0)
            print(f"Mensaje enviado a {topic}: {payload}")

        def subscribe(self, topic, callback):
            def wrapper(client, userdata, message):
                try:
                    payload_str = message.payload.decode()
                    payload = json.loads(payload_str)
                    if topic in [ "environmental/sensor/sub", "nutrient/solution/sensor/sub", "consumption/sensor/sub", "actuators/sub"]:
                        # process_sensor_message_sub(topic, payload)
                        self.loop.create_task(process_sensor_message_sub(topic, payload))
                    else:
                        self.loop.create_task(process_sensor_message_pub(message.topic, payload))
                except json.JSONDecodeError:
                    print("Error al decodificar el mensaje JSON.")
                except Exception as e:
                    print(f"Error inesperado en wrapper: {e}")
            
            self.client.subscribe(topic, 1, wrapper)
            print(f"Suscripto a {topic}")

        def unsubscribe(self, topic):
            self.client.unsubscribe(topic)
            print(f"Desuscripto de {topic}")

\end{lstlisting}

El código \ref{cod:integracion_fastapi} muestra la integración del cliente MQTT
con la aplicación en FastAPI. Se define un cliente MQTT y se suscribe a los
tópicos de publicación de datos de sensores y actuadores y tópicos de envío de
parámetros a sensores y actuadores. Además, se definen los endpoints para
probar la conexión MQTT y publicar mensajes desde la aplicación FastAPI.

\begin{lstlisting}[label=cod:integracion_fastapi,caption=Cliente MQTT en FastAPI., language=PythonUTF8]

# Fragmento de código para la integración del cliente MQTT con FastAPI

# Inicializar cliente MQTT
mqtt_client = AWSMQTTClient()

@app.on_event("startup")
async def startup():
    await init_db()
    mqtt_client.connect()
         
    # Suscripción a tópicos de publicación de sensores y actuadores
    mqtt_client.subscribe("environmental/sensor/pub", process_sensor_message_pub)
    mqtt_client.subscribe("nutrient_solution/sensor/pub", process_sensor_message_pub)
    mqtt_client.subscribe("consumption/sensor/pub", process_sensor_message_pub)
    mqtt_client.subscribe("actuators/pub", process_sensor_message_pub)

    # Suscripción a tópicos de envío de parámetros a sensores y actuadores
    mqtt_client.subscribe("environmental/sensor/sub", process_sensor_message_sub)
    mqtt_client.subscribe("nutrient_solution/sensor/sub", process_sensor_message_sub)
    mqtt_client.subscribe("consumption/sensor/sub", process_sensor_message_sub)
    mqtt_client.subscribe("actuators/sub", process_sensor_message_sub)

@app.on_event("shutdown")
async def shutdown():
    mqtt_client.disconnect()

# Clase para manejar la publicación de mensajes MQTT
class PublishRequest(BaseModel):
    topic: str
    message: dict
    
# Endpoint para publicar mensajes MQTT
@app.post("/mqtt/publish")
def publish_message(request: PublishRequest, user: dict = Depends(current_user)):
    mqtt_client.publish(request.topic, request.message)
    return {"status": "Mensaje publicado", "topic": request.topic, "message": request.message}
    
# Endpoint para probar la conexión MQTT
@app.get("/mqtt/test")
def test_mqtt_connection(user: dict = Depends(current_user)):
    try:
        mqtt_client.connect()
        return {"status": "Conexión exitosa"}
    except Exception as e:
        return {"status": "Error de conexión", "error": str(e)}
\end{lstlisting}

