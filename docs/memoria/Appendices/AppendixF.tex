\chapter{Conexión por WebSocket en FastAPI}
\label{AppendixF}

%%%%%%%%%%%%%%%%%%%%%%%%%%%%%%%%%%%%%%%%%%%%%%%%%%%%%%%%%%%%%%%%%%%%%%%%%%%%%
% parámetros para configurar el formato del código en los entornos lstlisting
%%%%%%%%%%%%%%%%%%%%%%%%%%%%%%%%%%%%%%%%%%%%%%%%%%%%%%%%%%%%%%%%%%%%%%%%%%%%%
\lstset{ %
    backgroundcolor=\color{white},   % choose the background color; you must add \usepackage{color} or \usepackage{xcolor}
    basicstyle=\footnotesize,        % the size of the fonts that are used for the code
    breakatwhitespace=false,         % sets if automatic breaks should only happen at whitespace
    breaklines=true,                 % sets automatic line breaking
    captionpos=b,                    % sets the caption-position to bottom
    commentstyle=\color{mygreen},    % comment style
    deletekeywords={...},            % if you want to delete keywords from the given language
    %escapeinside={\%*}{*)},          % if you want to add LaTeX within your code
    %extendedchars=true,              % lets you use non-ASCII characters; for 8-bits encodings only, does not work with UTF-8
    %frame=single,	                % adds a frame around the code
    keepspaces=true, keywordstyle=\color{blue}, language=[ANSI]C, % keeps spaces in text, useful for keeping indentation of code (possibly needs columns=flexible)% keyword style% the language of the code
    %otherkeywords={*,...},           % if you want to add more keywords to the set
    numbers=left, numbersep=5pt, numberstyle=\tiny\color{mygray},
    rulecolor=\color{black}, showspaces=false, showstringspaces=false,
    showtabs=false, stepnumber=1, stringstyle=\color{mymauve}, tabsize=2,
    title=\lstname, morecomment=[s]{/*}{*/} }% where to put the line-numbers; possible values are (none, left, right)% how far the line-numbers are from the code% the style that is used for the line-numbers% if not set, the frame-color may be changed on line-breaks within not-black text (e.g. comments (green here))% show spaces everywhere adding particular underscores; it overrides 'showstringspaces'% underline spaces within strings only% show tabs within strings adding particular underscores% the step between two line-numbers. If it's 1, each line will be numbered% string literal style% sets default tabsize to 2 spaces% show the filename of files included with \lstinputlisting; also try caption instead of title

\lstdefinelanguage{PythonUTF8}[]{Python}{
literate={á}{{\'a}}1 {é}{{\'e}}1 {í}{{\'i}}1 {ó}{{\'o}}1 {ú}{{\'u}}1
{Á}{{\'A}}1 {É}{{\'E}}1 {Í}{{\'I}}1 {Ó}{{\'O}}1 {Ú}{{\'U}}1
{ñ}{{\~n}}1 {Ñ}{{\~N}}1
}

\definecolor{mygreen}{rgb}{0,0.6,0}
\definecolor{mygray}{rgb}{0.5,0.5,0.5}
\definecolor{mymauve}{rgb}{0.58,0,0.82}

\section{Introducción}

Para establecer una conexión WebSocket se emplea el módulo \texttt{websockets}
de FastAPI, el cual permite crear un servidor que gestiona conexiones en tiempo
real con múltiples clientes. En este caso, el cliente WebSocket se conecta a la
ruta \texttt{/ws/sensor-data} para recibir información de sensores y
actuadores. La conexión permanece activa mientras se intercambian datos, y se
cierra automáticamente si ocurre un error o si el usuario no está autorizado.%``/ws/sensor-data'' 

\section{Autorización para WebSocket}

El código \ref{cod:web_socket_autorization} muestra la función encargada de
gestionar la autenticación del usuario a través del WebSocket. El token de
acceso puede enviarse en el encabezado \texttt{Authorization} o como
\texttt{Query Params}. Si el token es válido, se autentica al usuario; en caso
contrario, se cierra la conexión.

\begin{lstlisting}[label=cod:web_socket_autorization,caption=Autorización para WebSocket. , language=PythonUTF8]
from fastapi import WebSocket, status
from utils.authentication import auth_user, current_user
    
async def websocket_current_user(websocket: WebSocket):
    try:
        # Obtener el token del header o query parameter
        token = websocket.headers.get("authorization") or websocket.query_params.get("token")
        if not token:
            await websocket.close(code=status.WS_1008_POLICY_VIOLATION)
            return None
                
        # Limpiar el token (eliminar 'Bearer ' si existe)
        if token.startswith("Bearer "):
            token = token[7:]
                
        user = await auth_user(token)
        return await current_user(user)
           
    except Exception as e:
        print(f"Error en la autenticación del WebSocket: {e}")
        await websocket.send_text( f"Error en la autenticación del WebSocket: {str(e)}")
        await websocket.close(code=status.WS_1008_POLICY_VIOLATION)
        return None
\end{lstlisting}

\section{Gestión de conexiones WebSocket}

En el código \ref{cod:web_socket} se presenta la clase
\texttt{WebSocketManager}, encargada de gestionar las conexiones WebSocket.
Esta clase sigue el patrón Singleton y mantiene una lista de conexiones
activas, además de un caché con los últimos datos enviados por tipo de sensor.
También gestiona la conexión y desconexión de los clientes, el envío de datos
en tiempo real, y la actualización del caché con cada nuevo mensaje recibido.

\begin{lstlisting}[label=cod:web_socket,caption=Definición de clase WebSocket. , language=PythonUTF8]
import datetime
from typing import List, Dict, Optional
from fastapi import WebSocket 
    
class WebSocketManager:
    _instance: Optional['WebSocketManager'] = None
        
    # Método de clase para obtener la instancia única (Singleton)
    def __new__(cls):
        if cls._instance is None:
            cls._instance = super().__new__(cls)
            cls._instance.active_connections: List[Dict] = []
            cls._instance.sensor_data_cache: Dict[str, dict] = {
                "environmental": {},
                "nutrient_solution": {},
                "consumption": {},
                "actuators": {}
            }
        return cls._instance
        
    # Método para manejar la conexión de un nuevo cliente WebSocket
    async def connect(self, websocket: WebSocket, user: dict):
        self.active_connections.append({
            "websocket": websocket,
            "user": user
        })
        # Enviar datos almacenados al nuevo cliente
        for sensor_type, data in self.sensor_data_cache.items():
            if data:
                await websocket.send_json({
                    "type": sensor_type,
                    "data": data["data"],
                    "timestamp": data["timestamp"]
                })
        
    # Método para manejar la desconexión de un cliente WebSocket
    def disconnect(self, websocket: WebSocket):
        self.active_connections = [
            conn for conn in self.active_connections 
            if conn["websocket"] != websocket
        ]
        
    # Método para enviar un mensaje a un cliente WebSocket específico
    async def broadcast(self, message: dict):
        for connection in self.active_connections[:]:
            try:
                await connection["websocket"].send_json(message)
            except Exception as e:
                print(f"Error broadcasting to WebSocket: {e}")
                self.disconnect(connection)
        
    # Método para enviar un mensaje a todos los clientes WebSocket
    def update_cache(self, sensor_type: str, data: dict):
        self.sensor_data_cache[sensor_type] = {
            "data": data,
            "timestamp": datetime.datetime.now().isoformat()
        }
        
    # Método para obtener datos almacenados en caché
    def get_cached_data(self, sensor_type: str) -> dict:
        return self.sensor_data_cache.get(sensor_type, {})
        
    # Método para obtener la cantidad de clientes conectados
    @property
    def connected_clients(self) -> int:
        return len(self.active_connections)
    
# Instancia del WebSocketManager
websocket_manager = WebSocketManager()
\end{lstlisting}

\section{Integración con FastAPI}

Finalmente, el fragmento \ref{cod:integracion_fastapi} muestra cómo se integra
WebSocket en FastAPI. Cuando un cliente intenta conectarse a
\texttt{/ws/sensor-data}, se autentica al usuario, se acepta la conexión y se
agrega al gestor. La conexión se mantiene abierta mientras no se produzca un
error o una desconexión por parte del cliente. En ambos casos, el gestor
elimina al cliente de la lista de conexiones activas.

\begin{lstlisting}[label=cod:integracion_fastapi,caption=Cliente WebSocket en FastAPI., language=PythonUTF8]

# Fragmento de código para la integración del cliente MQTT con FastAPI

# Importamos el cliente WebSocket
from utils.websocket import websocket_manager
# Importamos el cliente WebSocket para autenticación
from utils.websocket_auth import websocket_current_user

# WebSocket para recibir datos de sensores y actuadores
@app.websocket("/ws/sensor-data")
async def websocket_endpoint(websocket: WebSocket):
    # Aceptar la conexión WebSocket
    await websocket.accept()
    # Obtener el usuario actual
    user = await websocket_current_user(websocket)
    # Verificar si el usuario está autenticado
    if not user:
        # Cerrar la conexión si no está autenticado
        return
    
    # Conectar el WebSocket al gestor
    await websocket_manager.connect(websocket, user)
    
    # Mantener la conexión abierta
    try:
        while True:
            await websocket.receive_text()
    except WebSocketDisconnect:
        websocket_manager.disconnect(websocket)
    except Exception as e:
        print(f"Error en WebSocket: {e}")
        websocket_manager.disconnect(websocket)
\end{lstlisting}

