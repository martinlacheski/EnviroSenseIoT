\chapter{Firmware del nodo sensor de consumos}
\label{AppendixG}

%%%%%%%%%%%%%%%%%%%%%%%%%%%%%%%%%%%%%%%%%%%%%%%%%%%%%%%%%%%%%%%%%%%%%%%%%%%%%
% parámetros para configurar el formato del código en los entornos lstlisting
%%%%%%%%%%%%%%%%%%%%%%%%%%%%%%%%%%%%%%%%%%%%%%%%%%%%%%%%%%%%%%%%%%%%%%%%%%%%%
\lstset{ %
    backgroundcolor=\color{white},   % choose the background color; you must add \usepackage{color} or \usepackage{xcolor}
    basicstyle=\footnotesize,        % the size of the fonts that are used for the code
    breakatwhitespace=false,         % sets if automatic breaks should only happen at whitespace
    breaklines=true,                 % sets automatic line breaking
    captionpos=b,                    % sets the caption-position to bottom
    commentstyle=\color{mygreen},    % comment style
    deletekeywords={...},            % if you want to delete keywords from the given language
    %escapeinside={\%*}{*)},          % if you want to add LaTeX within your code
    %extendedchars=true,              % lets you use non-ASCII characters; for 8-bits encodings only, does not work with UTF-8
    %frame=single,	                % adds a frame around the code
    keepspaces=true, keywordstyle=\color{blue}, language=[ANSI]C, % keeps spaces in text, useful for keeping indentation of code (possibly needs columns=flexible)% keyword style% the language of the code
    %otherkeywords={*,...},           % if you want to add more keywords to the set
    numbers=left, numbersep=5pt, numberstyle=\tiny\color{mygray},
    rulecolor=\color{black}, showspaces=false, showstringspaces=false,
    showtabs=false, stepnumber=1, stringstyle=\color{mymauve}, tabsize=2,
    title=\lstname, morecomment=[s]{/*}{*/} }% where to put the line-numbers; possible values are (none, left, right)% how far the line-numbers are from the code% the style that is used for the line-numbers% if not set, the frame-color may be changed on line-breaks within not-black text (e.g. comments (green here))% show spaces everywhere adding particular underscores; it overrides 'showstringspaces'% underline spaces within strings only% show tabs within strings adding particular underscores% the step between two line-numbers. If it's 1, each line will be numbered% string literal style% sets default tabsize to 2 spaces% show the filename of files included with \lstinputlisting; also try caption instead of title

\lstdefinelanguage{PythonUTF8}[]{Python}{
literate={á}{{\'a}}1 {é}{{\'e}}1 {í}{{\'i}}1 {ó}{{\'o}}1 {ú}{{\'u}}1
{Á}{{\'A}}1 {É}{{\'E}}1 {Í}{{\'I}}1 {Ó}{{\'O}}1 {Ú}{{\'U}}1
{ñ}{{\~n}}1 {Ñ}{{\~N}}1
}

\definecolor{mygreen}{rgb}{0,0.6,0}
\definecolor{mygray}{rgb}{0.5,0.5,0.5}
\definecolor{mymauve}{rgb}{0.58,0,0.82}

\section{Introducción}


El código \ref{cod:firmware} muestra la implementación del firmware para el nodo
sensor de consumos. Este firmware se encarga de gestionar la conexión
del nodo con el broker MQTT.

El código está diseñado para ejecutarse en un microcontrolador ESP32 y utiliza
librerías específicas para manejar sensores y la conexión a Wi-Fi.



\begin{lstlisting}[label=cod:firmware,caption=Firmware nodo sensor de consumos. , language=PythonUTF8]
    from machine import Pin, Timer, reset, UART
    from lib.pzem import PZEM
    from lib.hcsr04 import HCSR04
    from lib.wifi_manager import WifiManager
    from lib.robust import MQTTClient
    import config
    import ntptime
    import ssl
    import time
    import uos
    import gc
    import sys
    import json
    
    # Configuración inicial
    CONFIG_FILE = "interval.conf"
    WIFI_FILE = "wifi.dat"
    TIMEZONE_FILE = "timezone.conf"
    DEFAULT_INTERVAL = 5
    sensor_interval = DEFAULT_INTERVAL
    sensor_data = {}  # Diccionario para datos de sensores
    wm = WifiManager()
    
    # Variables globales
    wlan = None
    wifi_connected = False
    last_led_toggle = time.ticks_ms()
    led_state = False
    mqtt_client = None
    last_wifi_check = 0
    WIFI_CHECK_INTERVAL = 10000  # 10 segundos
    
    # Configurar botón BOOT (GPIO0) con pull-up
    boot_button = Pin(0, Pin.IN, Pin.PULL_UP)
    led = Pin(2, Pin.OUT) # LED azul en GPIO2 (común en ESP32)
    
    # Inicialización de sensores
    print("Inicializando sensores")
    
    # UART PZEM-004T: TX- GPIO25, RX - GPIO26 
    uart = UART(1, baudrate=9600, tx=25, rx=26)
    
    # Crear instancia del PZEM
    try:
        pzem = PZEM(uart)
        print("Sensor PZEM-004-T inicializado")
    except Exception as e:
        print("Error inicializando sensor PZEM-004-T:", e)
    
    # Configuración pines HCSR04
    # Sensor 1: Trig - GPIO13, Echo - GPIO12
    # Sensor 2: Trig - GPIO14, Echo - GPIO27
    # Sensor 3: Trig - GPIO16, Echo - GPIO17
    # Sensor 4: Trig - GPIO18, Echo - GPIO19
    # Sensor 5: Trig - GPIO21, Echo - GPIO22
    # Sensor 6: Trig - GPIO32, Echo - GPIO33
    
    # Sensor de distancia HCSR04-1
    try:
        level_1 = HCSR04(trigger_pin=13, echo_pin=12, echo_timeout_us=10000)
        print("Sensor HCSR04-1 inicializado")
    except Exception as e:
        print("Error inicializando sensor HCSR04-1:", e)
    
    # Sensor de distancia HCSR04-2
    try:
        level_2 = HCSR04(trigger_pin=14, echo_pin=27, echo_timeout_us=10000)
        print("Sensor HCSR04-2 inicializado")
    except Exception as e:
        print("Error inicializando sensor HCSR04-2:", e)
        
    # Sensor de distancia HCSR04-3
    try:
        level_3 = HCSR04(trigger_pin=16, echo_pin=17, echo_timeout_us=10000)
        print("Sensor HCSR04-3 inicializado")
    except Exception as e:
        print("Error inicializando sensor HCSR04-3:", e)
        
    # Sensor de distancia HCSR04-4
    try:
        level_4 = HCSR04(trigger_pin=18, echo_pin=19, echo_timeout_us=10000)
        print("Sensor HCSR04-4 inicializado")
    except Exception as e:
        print("Error inicializando sensor HCSR04-4:", e)
        
    # Sensor de distancia HCSR04-5
    try:
        level_5 = HCSR04(trigger_pin=21, echo_pin=22, echo_timeout_us=10000)
        print("Sensor HCSR04-5 inicializado")
    except Exception as e:
        print("Error inicializando sensor HCSR04-5:", e)
    
    # Sensor de distancia HCSR04-6
    try:
        level_6 = HCSR04(trigger_pin=32, echo_pin=33, echo_timeout_us=10000)
        print("Sensor HCSR04-6 inicializado")
    except Exception as e:
        print("Error inicializando sensor HCSR04-6:", e)
        
    # Liberamos la memoria
    gc.collect()
        
    # Método para conectar Wi-Fi con mejor manejo de errores
    def connect_wifi():
        global wifi_connected
        try:
            if not wm.is_connected():
                print("Intentando conectar red Wi-Fi...")
                wlan = wm.connect()
                if wm.is_connected():
                    wifi_connected = True
                    led.on()
                    return True  
            wifi_connected = False
            led.off()
            return False
        except Exception as e:
            print("Error en la conexión de Wi-Fi:", e)
            wifi_connected = False
            led.off()
            return False
    
    # Método para verificar estado del WiFi
    def check_wifi_connection():
        
        global wifi_connected, last_wifi_check
        current_time = time.ticks_ms()
        
        if time.ticks_diff(current_time, last_wifi_check) > WIFI_CHECK_INTERVAL:
            last_wifi_check = current_time
            
            if not wm.is_connected():
                print("Wi-Fi desconectado. Intentando reconectar...")
                if connect_wifi():
                    wifi_connected = True
                    return True
                else:
                    wifi_connected = False
                    return False
            
            wifi_connected = True
        return wifi_connected
    
    # Método para sincronizar tiempo con NTP
    def sync_time(max_retries=5):
        for i in range(max_retries):
            try:
                print(f"Intentando sincronizar hora (intento {i+1}/{max_retries})...")
                ntptime.host = "time.google.com"  # Servidor alternativo
                ntptime.settime()
                print("Hora sincronizada:", time.localtime())
                return True
            except OSError as e:
                print("Error sincronizando hora:", e)
                time.sleep(2)
        print("Error: No se pudo sincronizar la hora después de", max_retries, "intentos")
        return False
    
    # Carga de certificados
    try:
        print("Cargando certificados...")
        with open("aws/client.key", "rb") as f:
            CLIENT_KEY = f.read()
        with open("aws/client.crt", "rb") as f:
            CLIENT_CRT = f.read()
        with open("aws/root.crt", "rb") as f:
            ROOT_CRT = f.read()
        print("Certificados cargados correctamente")
    except Exception as e:
        print("Error cargando certificados:", e)
        raise
    
    #Leer la zona horaria desde el archivo de configuración
    #TIMEZONE_FILE = "timezone.conf"
    try:
        with open(TIMEZONE_FILE, 'r') as f:
            TIMEZONE = f.read().strip()
            print("Zona horaria configurada:", TIMEZONE)
    except Exception as e:
        print("Error leyendo zona horaria:", e)
        raise
    
    # Liberamos la memoria
    gc.collect()
    
    # Configuración de conexión MQTT
    try:
        print("Configurando SSL...")
        context = ssl.SSLContext(ssl.PROTOCOL_TLS_CLIENT)
        context.verify_mode = ssl.CERT_REQUIRED
        context.load_cert_chain(CLIENT_CRT, CLIENT_KEY)
        context.load_verify_locations(cadata=ROOT_CRT)
        print("SSL configurado correctamente")
    except Exception as e:
        print("Error configurando SSL:", e)
        raise
    
    # Creación de cliente MQTT
    try:
        print("Creando cliente MQTT...")
        mqtt_client = MQTTClient(
            client_id=config.AWS_CLIENT_ID,
            server=config.AWS_ENDPOINT,
            port=8883,
            keepalive=5000,
            ssl=context,
        )
        print("Cliente MQTT creado")
    except Exception as e:
        print("Error creando cliente MQTT:", e)
        raise
    
    # Callback para mensajes entrantes (Seteo de nuevo intervalo y lectura inmediata)
    def subscription_cb(topic, message):
        print("\nMensaje recibido:")
        print("Tópico:", topic.decode("utf-8"))
        print("Mensaje:", message.decode("utf-8"))
        print("-------------------")
        
        try:
            msg = json.loads(message.decode("utf-8"))
            
            # Verificar si el mensaje es para este sensor
            if msg.get("sensor_code") != config.SENSOR_CODE:
                print("Mensaje no destinado a este sensor")
                return
            
            # Comando para lectura inmediata
            if msg.get("command") == "read_now":
                print("Comando de lectura inmediata recibido")
                
                # Enviar confirmación
                response = {
                    "sensor_code": config.SENSOR_CODE,
                    "command": "read_now_ack",
                    "status": "received"
                }
                mqtt_client.publish(config.AWS_TOPIC_SUB, json.dumps(response), qos=0)
                print("Confirmación de lectura enviada al servidor")
                
                # Ejecutar lectura inmediata
                leer_sensores()
                
            # Comando para lectura inmediata
            if msg.get("command") == "read_now_ack":
                pass
            
            # Cambio de intervalo
            elif "interval" in msg:
                new_interval = msg.get("interval")
                
                # Validar intervalo
                if isinstance(new_interval, int) and 1 <= new_interval <= 86400:
                    global sensor_interval, timer
                    
                    # Actualizar y guardar intervalo
                    sensor_interval = new_interval
                    if save_interval(sensor_interval):
                        # Reconfigurar timer
                        timer.deinit()
                        timer.init(period=sensor_interval*1000, mode=Timer.PERIODIC, callback=lambda t: leer_sensores())
                        print("Intervalo actualizado:", sensor_interval)
                        
                        # Enviar confirmación
                        response = {
                            "sensor_code": config.SENSOR_CODE,
                            "interval": "OK",
                            "seconds_to_report": sensor_interval
                        }
                        mqtt_client.publish(config.AWS_TOPIC_SUB, json.dumps(response), qos=0)
                        print("Confirmación enviada al servidor")
            
            # Comando no reconocido
            #else:
            #    print("Comando no reconocido en el mensaje", msg)
                    
        except Exception as e:
            print("Error procesando mensaje:", e)
            
        # Liberamos la memoria
        gc.collect()
    
    # Método para reiniciar el dispositivo 
    def check_boot_button():
        try:
            if boot_button.value() == 0:  
                time.sleep(0.1)  # Esperar 100ms para confirmar pulsación
                if boot_button.value() == 0:  # Si sigue presionado, iniciar conteo
                    print("\nBotón BOOT detectado - Iniciando conteo...")
                    start_time = time.ticks_ms()
                    pressed = True
                    while time.ticks_diff(time.ticks_ms(), start_time) < 3000:
                        if boot_button.value() == 1:
                            pressed = False
                            break
                        time.sleep_ms(100)
                    
                    if pressed:
                        print("\n--- RESETEO DE CONFIGURACIÓN ---")
                        try:
                            uos.remove(WIFI_FILE)
                            print(f"Archivo {WIFI_FILE} eliminado")
                        except OSError as e:
                            print(f"Error eliminando el archivo: {e}")
                        
                        print("Reiniciando dispositivo...\n")
                        time.sleep(1)
                        reset()
                    else:
                        print("Reset cancelado")
            return False
        except Exception as e:
            print("Error en check_boot_button:", e)
            
    # Cargar y guardar intervalo de envío de datos
    def load_interval():
        try:
            with open(CONFIG_FILE, 'r') as f:
                interval = int(f.read())
                print(f"Intervalo de envío de datos configurado en: {interval} segundos")
                return interval
        except:
            print(f"No se pudo cargar el intervalo, usando valor por defecto: {DEFAULT_INTERVAL} segundos")
            return DEFAULT_INTERVAL
    
    # Guardar intervalo de envío de datos
    def save_interval(value):
        try:
            with open(CONFIG_FILE, 'w') as f:
                f.write(str(value))
            print(f"Intervalo guardado: {value} segundos")
            return True
        except Exception as e:
            print(f"Error guardando intervalo: {e}")
            return False
    
    # Leer sensores y enviar datos por MQTT
    def leer_sensores(new_interval=None):        
        global sensor_interval, timer
        
        try:
            if new_interval is not None and 1 <= new_interval <= 86400:
                sensor_interval = new_interval
                save_interval(sensor_interval)
                timer.deinit()
                timer.init(period=sensor_interval*1000, mode=Timer.PERIODIC, callback=lambda t: leer_sensores())
        except Exception as e:
            print("Error en new_interval:", e)
    
        try:
            # Verificación silenciosa de WiFi
            if not check_wifi_connection():
                return
            
            # Obtener la hora local ajustada por timezone
            def adjust_time_with_timezone(utc_time, timezone_offset):
                """Ajusta la hora UTC según el offset de timezone (ej: '-03:00')"""
                try:
                    # Parsear el offset de timezone
                    sign = -1 if timezone_offset[0] == '-' else 1
                    hours = int(timezone_offset[1:3])
                    minutes = int(timezone_offset[4:6])
                    total_offset = sign * (hours * 3600 + minutes * 60)
                    
                    # Convertir tiempo local a segundos desde epoch
                    epoch_time = time.mktime(utc_time)
                    
                    # Aplicar el offset
                    adjusted_time = epoch_time + total_offset
                    return time.localtime(adjusted_time)
                except Exception as e:
                    print("Error ajustando zona horaria:", e)
                    return utc_time  # Si hay error, devolver la hora sin ajuste
            
            # Se toma la fecha y hora del ESP32 y se ajusta por timezone
            current_time = time.localtime()
            adjusted_time = adjust_time_with_timezone(current_time, TIMEZONE)
            # print("La hora ajustada segun zona horaria es: ", adjusted_time)
            
            # Formateamos la fecha y hora
            year, month, day, hour, minute, second = adjusted_time[:6]
            fecha_formateada = f"{year}-{month:02d}-{day:02d} {hour:02d}:{minute:02d}:{second:02d}"
            
            # Se acceden a los sensores
            
            # Leer PZEM-004-T
            if pzem.read():
                voltage = pzem.getVoltage()
                current = pzem.getCurrent()
                power = pzem.getActivePower()
                energy = pzem.getActiveEnergy()
                frecuency = pzem.getFrequency()
                power_factor = pzem.getPowerFactor()
            else:
                    print("Error al leer los datos del PZEM")
            
            # Leer niveles
            distance_level_1 = round(level_1.distance_cm(),2)
            distance_level_2 = round(level_2.distance_cm(),2)
            distance_level_3 = round(level_3.distance_cm(),2)
            distance_level_4 = round(level_4.distance_cm(),2)
            distance_level_5 = round(level_5.distance_cm(),2)
            distance_level_6 = round(level_6.distance_cm(),2)
            
            try:
                sensor_data.update({
                    "sensor_code": config.SENSOR_CODE,
                    "voltage": round(voltage,2),
                    "current": round(current,2),
                    "power": round(power,2),
                    "energy": energy,
                    "frecuency": round(frecuency,2),
                    "power_factor": round(power_factor,2),
                    "nutrient_1_level": distance_level_1,
                    "nutrient_2_level": distance_level_2,
                    "nutrient_3_level": distance_level_3,
                    "nutrient_4_level": distance_level_4,
                    "nutrient_5_level": distance_level_5,
                    "nutrient_6_level": distance_level_6,
                    "datetime": fecha_formateada,
                })
            except Exception as e:
                print("Error en sensor_data.update:", e)
    
            # Enviar por MQTT si está conectado
            try:
                mqtt_client.publish(topic=config.AWS_TOPIC_PUB, msg=json.dumps(sensor_data), qos=0)
                print("Mensaje publicado: ", json.dumps(sensor_data))
            except Exception as e:
                print("Error en cliente MQTT, reconectando...")
                mqtt_client.connect()
                mqtt_client.publish(topic=config.AWS_TOPIC_PUB, msg=json.dumps(sensor_data), qos=0)
                
            # Liberamos la memoria
            gc.collect()
                
        except Exception as e:
            print("Error en check_wifi_connection:", e)
    
    # Código principal
    try:
        print("\nIniciando dispositivo...")
        led.off()  # Comenzar con LED apagado
        
        # Intento inicial de conexión WiFi
        try:
            if not connect_wifi():
                print("No se pudo conectar al Wi-Fi en el inicio")
            else:
                # Sincronizar hora si WiFi está conectado
                if not sync_time():
                    print("Advertencia: No se pudo sincronizar la hora por NTP")
                
                # Liberamos la memoria
                gc.collect()
                
                # Conexión al servidor AWS IoT Core
                print("Intentando conectar a AWS IoT Core...")
                try:
                    mqtt_client.connect()
                    print("Conectado a AWS IoT Core correctamente")
                    
                    # Suscribirse a topic de MQTT
                    try:
                        mqtt_client.set_callback(subscription_cb)
                        mqtt_client.subscribe(config.AWS_TOPIC_SUB)
                        print(f"Suscrito al tópico: {config.AWS_TOPIC_SUB}")
                    except Exception as e:
                        print("Error al suscribirse:", e)
                except Exception as e:
                    print("Error conectando a AWS IoT Core:", e)
        except Exception as e:
            print("Error en connect_wifi:", e)
        # Configuración inicial
        sensor_interval = load_interval()
        
        # Configurar timer y leer sensores y liberar memoria
        timer = Timer(-1)
        # Liberamos la memoria
        gc.collect()
        # Tarea de lectura de sensores
        timer.init(period=sensor_interval*1000, mode=Timer.PERIODIC, callback=lambda t: leer_sensores())
        # Liberamos la memoria
        gc.collect()
            
        # Bucle principal
        while True:
            # Liberamos la memoria
            gc.collect()
            try:
                # Verificación periódica silenciosa
                check_wifi_connection()
                            
                # Verificar botón de reset
                check_boot_button()
                    
                # Chequear mensajes MQTT
                if wifi_connected:
                    try:
                        mqtt_client.check_msg()
                    except Exception as e:
                        print("Error en el cliente MQTT:", e)
                
                # Pequeño delay para no saturar el procesador
                time.sleep(0.1)
                
            except Exception as e:
                print("Error crítico:", e)
                time.sleep(5)
    
    except KeyboardInterrupt:
        print("\nPrograma detenido por el usuario")
        try:
            timer.deinit()
        except:
            pass
        led.off()
        reset()
    
    except Exception as e:
        print("Error fatal:", e)
        sys.print_exception(e)
        time.sleep(2)
        reset()
\end{lstlisting}