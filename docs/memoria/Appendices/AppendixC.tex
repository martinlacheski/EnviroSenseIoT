\chapter{Conexión al broker MQTT con FastAPI y AWS IoT SDK para Python}
\label{AppendixC}

La conexión al broker MQTT se realiza para permitir la comunicación entre el
servidor IoT y el broker MQTT.

El código \ref{cod:mqtt_connection} muestra el proceso de conexión a AWS IoT
Core y la implementación de la lógica de publicación y suscripción a los
tópicos.

En este código, se definen los métodos para conectar al broker, publicar y
suscribirse a tópicos, y manejar los mensajes recibidos. Se implementó un
cliente MQTT que interactúa con AWS IoT Core, gestionando la comunicación con
los nodos sensores y actuadores. Además, se incorporaron métodos para recibir
datos de estos dispositivos, enviarles comandos y almacenar la información en
la base de datos MongoDB.

%%%%%%%%%%%%%%%%%%%%%%%%%%%%%%%%%%%%%%%%%%%%%%%%%%%%%%%%%%%%%%%%%%%%%%%%%%%%%
% parámetros para configurar el formato del código en los entornos lstlisting
%%%%%%%%%%%%%%%%%%%%%%%%%%%%%%%%%%%%%%%%%%%%%%%%%%%%%%%%%%%%%%%%%%%%%%%%%%%%%
\lstset{ %
    backgroundcolor=\color{white},   % choose the background color; you must add \usepackage{color} or \usepackage{xcolor}
    basicstyle=\footnotesize,        % the size of the fonts that are used for the code
    breakatwhitespace=false,         % sets if automatic breaks should only happen at whitespace
    breaklines=true,                 % sets automatic line breaking
    captionpos=b,                    % sets the caption-position to bottom
    commentstyle=\color{mygreen},    % comment style
    deletekeywords={...},            % if you want to delete keywords from the given language
    %escapeinside={\%*}{*)},          % if you want to add LaTeX within your code
    %extendedchars=true,              % lets you use non-ASCII characters; for 8-bits encodings only, does not work with UTF-8
    %frame=single,	                % adds a frame around the code
    keepspaces=true, keywordstyle=\color{blue}, language=[ANSI]C, % keeps spaces in text, useful for keeping indentation of code (possibly needs columns=flexible)% keyword style% the language of the code
    %otherkeywords={*,...},           % if you want to add more keywords to the set
    numbers=left, numbersep=5pt, numberstyle=\tiny\color{mygray},
    rulecolor=\color{black}, showspaces=false, showstringspaces=false,
    showtabs=false, stepnumber=1, stringstyle=\color{mymauve}, tabsize=2,
    title=\lstname, morecomment=[s]{/*}{*/} }% where to put the line-numbers; possible values are (none, left, right)% how far the line-numbers are from the code% the style that is used for the line-numbers% if not set, the frame-color may be changed on line-breaks within not-black text (e.g. comments (green here))% show spaces everywhere adding particular underscores; it overrides 'showstringspaces'% underline spaces within strings only% show tabs within strings adding particular underscores% the step between two line-numbers. If it's 1, each line will be numbered% string literal style% sets default tabsize to 2 spaces% show the filename of files included with \lstinputlisting; also try caption instead of title

\lstdefinelanguage{PythonUTF8}[]{Python}{
literate={á}{{\'a}}1 {é}{{\'e}}1 {í}{{\'i}}1 {ó}{{\'o}}1 {ú}{{\'u}}1
{Á}{{\'A}}1 {É}{{\'E}}1 {Í}{{\'I}}1 {Ó}{{\'O}}1 {Ú}{{\'U}}1
{ñ}{{\~n}}1 {Ñ}{{\~N}}1
}

\definecolor{mygreen}{rgb}{0,0.6,0}
\definecolor{mygray}{rgb}{0.5,0.5,0.5}
\definecolor{mymauve}{rgb}{0.58,0,0.82}

\begin{lstlisting}[label=cod:mqtt_connection,caption=Cliente MQTT, language=PythonUTF8]
    import asyncio
    import json
    import os
    from AWSIoTPythonSDK.MQTTLib import AWSIoTMQTTClient
    from models.sensor_environmental_data import EnvironmentalSensorData
    from mqtt.aws_config import AWS_ENDPOINT, MQTT_CLIENT_ID
    
    # Método asíncrono para insertar datos en la base de datos
    async def insert_sensor_data(payload):
        try:
            sensor_data = EnvironmentalSensorData(**payload)
            await sensor_data.insert()
            print("Datos guardados en la base de datos.")
        except Exception as e:
            print("Error al guardar datos en la base de datos:", e)
    
    # Método para procesar mensajes de sensores y actuadores
    async def process_sensor_message_pub(topic, payload):
        try:
            print(f" Mensaje recibido desde {topic}")
            print(f"Data: {payload}")
    
            if topic in [
                "environmental/sensor/pub",
                "nutrient_solution/sensor/pub",
                "consumption/sensor/pub",
                "actuators/pub",
            ]:
                print(f"Procesando mensaje de {topic}")
                await insert_sensor_data(payload)
        except Exception as e:
            print(f"Error inesperado: {e}")
    
    # Método para procesar mensajes de control de sensores
    def process_sensor_message_sub(topic, payload):
        try:
            sensor_code = payload.get("sensor_code")
            interval = payload.get("interval")
    
            print(f"Mensaje recibido desde {topic}")
            print(f"Sensor: {sensor_code}, Intervalo: {interval}")
    
            if str(interval) == "OK":
                print("Nuevo intervalo de tiempo establecido")
            # else:
            #     print("rror al establecer el intervalo de tiempo.")
        except Exception as e:
            print(f"Error inesperado: {e}")
    
    class AWSMQTTClient:
        def __init__(self):
            CERTS_DIR = os.getenv("CERTS_DIR", "mqtt/certificates")
            self.ROOT_CRT = os.path.join(CERTS_DIR, "root.crt")
            self.CLIENT_CRT = os.path.join(CERTS_DIR, "client.crt")
            self.CLIENT_KEY = os.path.join(CERTS_DIR, "client.key")
    
            # Verificar existencia de certificados
            for cert in [self.ROOT_CRT, self.CLIENT_CRT, self.CLIENT_KEY]:
                if not os.path.exists(cert):
                    raise FileNotFoundError(f"El archivo {cert} no fue encontrado.")
    
            # Configurar cliente MQTT
            self.client = AWSIoTMQTTClient(MQTT_CLIENT_ID)
            self.client.configureEndpoint(AWS_ENDPOINT, 8883)
            self.client.configureCredentials(self.ROOT_CRT, self.CLIENT_KEY, self.CLIENT_CRT)
            self.client.configureConnectDisconnectTimeout(10)
            self.loop = asyncio.get_event_loop()
    
        def connect(self):
            print("Estableciendo conexión con AWS IoT Core.")
            self.client.connect()
            print(" Cliente MQTT Conectado.")
    
        def disconnect(self):
            self.client.disconnect()
            print("Cliente Desconectado")
    
        def publish(self, topic, message):
            payload = json.dumps(message) if isinstance(message, dict) else str(message)
            self.client.publish(topic, payload, 0)
            print(f"Mensaje enviado a {topic}: {payload}")
    
        def subscribe(self, topic, callback):
            def wrapper(client, userdata, message):
                try:
                    payload_str = message.payload.decode()
                    payload = json.loads(payload_str)
                    if topic in [ "environmental/sensor/sub", 
                            "nutrient_solution/sensor/sub", 
                            "consumption/sensor/sub", 
                            "actuators/sub"]:
                        process_sensor_message_sub(topic, payload)
                    else:
                        self.loop.create_task(process_sensor_message_pub(message.topic, payload))
                except json.JSONDecodeError:
                    print("Error al decodificar el mensaje JSON.")
                except Exception as e:
                    print(f"Error inesperado en wrapper: {e}")
            
            self.client.subscribe(topic, 1, wrapper)
            print(f"Suscripto a {topic}")
    
        def unsubscribe(self, topic):
            self.client.unsubscribe(topic)
            print(f"Desuscripto de {topic}")    
        \end{lstlisting}