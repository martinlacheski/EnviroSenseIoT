\chapter{Autenticación con JWT}

\label{AppendixB}
El control de acceso se realiza mediante la verificación de un token JWT que se
envía en las solicitudes. Si el token es válido, se permite el acceso a los
recursos protegidos; de lo contrario, se devuelve un error de autenticación.

El código utiliza la librería \texttt{passlib} para el manejo de contraseñas
y \texttt{bcrypt} para el cifrado. Además, se utiliza \texttt{fastapi.security}
para manejar la autenticación y autorización.

El siguiente código es un ejemplo de cómo implementar el control de acceso en
una API REST utilizando FastAPI y JWT.

%%%%%%%%%%%%%%%%%%%%%%%%%%%%%%%%%%%%%%%%%%%%%%%%%%%%%%%%%%%%%%%%%%%%%%%%%%%%%
% parámetros para configurar el formato del código en los entornos lstlisting
%%%%%%%%%%%%%%%%%%%%%%%%%%%%%%%%%%%%%%%%%%%%%%%%%%%%%%%%%%%%%%%%%%%%%%%%%%%%%
\lstset{ %
  backgroundcolor=\color{white},   % choose the background color; you must add \usepackage{color} or \usepackage{xcolor}
  basicstyle=\footnotesize,        % the size of the fonts that are used for the code
  breakatwhitespace=false,         % sets if automatic breaks should only happen at whitespace
  breaklines=true,                 % sets automatic line breaking
  captionpos=b,                    % sets the caption-position to bottom
  commentstyle=\color{mygreen},    % comment style
  deletekeywords={...},            % if you want to delete keywords from the given language
  %escapeinside={\%*}{*)},          % if you want to add LaTeX within your code
  %extendedchars=true,              % lets you use non-ASCII characters; for 8-bits encodings only, does not work with UTF-8
  %frame=single,	                % adds a frame around the code
  keepspaces=true, keywordstyle=\color{blue}, language=[ANSI]C, % keeps spaces in text, useful for keeping indentation of code (possibly needs columns=flexible)% keyword style% the language of the code
  %otherkeywords={*,...},           % if you want to add more keywords to the set
  numbers=left, numbersep=5pt, numberstyle=\tiny\color{mygray},
  rulecolor=\color{black}, showspaces=false, showstringspaces=false,
  showtabs=false, stepnumber=1, stringstyle=\color{mymauve}, tabsize=2,
  title=\lstname, morecomment=[s]{/*}{*/} }% where to put the line-numbers; possible values are (none, left, right)% how far the line-numbers are from the code% the style that is used for the line-numbers% if not set, the frame-color may be changed on line-breaks within not-black text (e.g. comments (green here))% show spaces everywhere adding particular underscores; it overrides 'showstringspaces'% underline spaces within strings only% show tabs within strings adding particular underscores% the step between two line-numbers. If it's 1, each line will be numbered% string literal style% sets default tabsize to 2 spaces% show the filename of files included with \lstinputlisting; also try caption instead of title

\lstdefinelanguage{PythonUTF8}[]{Python}{
literate={á}{{\'a}}1 {é}{{\'e}}1 {í}{{\'i}}1 {ó}{{\'o}}1 {ú}{{\'u}}1
{Á}{{\'A}}1 {É}{{\'E}}1 {Í}{{\'I}}1 {Ó}{{\'O}}1 {Ú}{{\'U}}1
{ñ}{{\~n}}1 {Ñ}{{\~N}}1
}

\definecolor{mygreen}{rgb}{0,0.6,0}
\definecolor{mygray}{rgb}{0.5,0.5,0.5}
\definecolor{mymauve}{rgb}{0.58,0,0.82}

\begin{lstlisting}[label=cod:vControl,caption=Pseudocódigo del control de acceso, language=PythonUTF8]
from fastapi import APIRouter, Depends, HTTPException, status
from fastapi.security import OAuth2PasswordBearer
from passlib.context import CryptContext
import jwt
import bcrypt
from models.user import User
KEY = "colocar_clave_secreta_aquí"
ALG = "HS256"
ACCESS_TOKEN_EXPIRE_MINUTES = 30
oauth2 = OAuth2PasswordBearer(tokenUrl="login")
crypt = CryptContext(schemes=["bcrypt"], deprecated="auto")
async def auth_user(token: str = Depends(oauth2)):
  exception = HTTPException(
    status_code=status.HTTP_401_UNAUTHORIZED,
    detail="Credenciales de autenticacion inválidas",
    headers={"WWW-Authenticate": "Bearer"},
  )
  try:
    user = jwt.decode(token, KEY, algorithms=[ALG]).get("username") 
    if user is None:
      raise exception   
  except:
    raise exception
  user = await User.find_one({"username": username})
  if not user:
    raise exception
  return user
  async def current_user(user: User = Depends(auth_user)):
    if not user.enabled:
      raise HTTPException(
        status_code=status.HTTP_400_BAD_REQUEST, detail="Usuario deshabilitado"
    )
    return user
\end{lstlisting}