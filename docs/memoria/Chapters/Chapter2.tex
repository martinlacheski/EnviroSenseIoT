\chapter{Introducción específica}

En este capítulo se presentan los protocolos de comunicación, componentes de
hardware y herramientas de software utilizados en el desarrollo del trabajo. Se
detallan las características y sus especificaciones técnicas.

%----------------------------------------------------------------------------------------
%	SECTION 1 - Protocolos de comunicación
%----------------------------------------------------------------------------------------

\section{Protocolos de comunicación}

En esta sección se describen los diferentes protocolos de comunicación
utilizados en el desarrollo del trabajo. % La tabla \ref{tab:comunicacion}
% presenta los principales protocolos utilizados y su función.

% \begin{table}[h]
% 	\centering
% 	\caption[Principales protocolos utilizados]{Principales protocolos utilizados}
% 	\begin{tabular}{p{3.2cm}p{9.6cm}}
% 		\toprule
% 		\textbf{Protocolo}         & \textbf{Función}                                                     \\
% 		\midrule
% 		\multirow{1}{*}{Wi-Fi}     & Conexión entre dispositivos e Internet.                              \\
% 		\multirow{1}{*}{MQTT}      & Protocolo ligero para mensajería entre dispositivos y servidor IoT.  \\
% 		\multirow{1}{*}{TLS}       & Protocolo de seguridad que cifra la comunicación entre dispositivos. \\
% 		\multirow{1}{*}{HTTP}      & Protocolo cliente/servidor utilizado entre backend y frontend.       \\
% 		\multirow{1}{*}{WebSocket} & Protocolo de comunicación entre cliente y servidor.                  \\
% 		\bottomrule
% 		\hline
% 	\end{tabular}
% 	\label{tab:comunicacion}
% \end{table}

\subsection{Wi-Fi}

Wi-Fi es el nombre comercial propiedad de la Wi-Fi Alliance para designar su
familia de protocolos de comunicación inalámbrica basados en el estándar IEEE
802.11 para redes de área local sin cables \cite{Li2019}.

El estandar identifica dos modos principales de topología de red:
infraestructura y ad-hoc.

\begin{itemize}
	\item Modo infraestructura: los dispositivos se conectan a una red inalámbrica a
	      través de un router o AP (\textit{Access Point}) inalámbrico, como en las WLAN.
	      Los AP se conectan a la infraestructura de la red mediante el sistema de
	      distribución conectado por cable o de manera inalámbrica.
	\item Modo ad-hoc: los dispositivos se conectan directamente entre sí sin necesidad
	      de un punto de acceso.
\end{itemize}

\subsection{MQTT}

MQTT es un protocolo de mensajería estándar internacional OASIS
\cite{OASIS_MQTT_Standard} para el Internet de las Cosas (IoT). Está diseñado
como un transporte de mensajería de publicación/suscripción extremadamente
ligero, ideal para conectar dispositivos remotos con un consumo de código
reducido y un ancho de banda de red mínimo.

MQTT es un protocolo ligero basado en TCP/IP \cite{AWS_MQTT} que sigue un
modelo de publicación/suscripción, donde:

\begin{itemize}
	\item Broker: Actúa como servidor central, que actuando como intermediario entre los
	      clientes, recibe todos los mensajes y los enruta a los clientes suscritos.
	\item Cliente: Puede ser un dispositivo que envía mensajes (publicación en un tópico)
	      o un dispositivo que recibe mensajes (suscripción a un tópico).
	\item Tópico: Es la dirección a la que se envían los mensajes. El broker MQTT es
	      responsable de enrutar los mensajes a los clientes suscritos a ese tema. Los
	      temas son organizados en tópicos.
\end{itemize}

La figura \ref{fig:MqttProtocol} muestra la arquitectura del protocolo MQTT.

\begin{figure}[H]
	\centering
	\includegraphics[width=.70\textwidth]{./Images/2.png}
	\caption{Arquitectura del protocolo MQTT.}
	\label{fig:MqttProtocol}
\end{figure}

\subsection{TLS}

TLS es un protocolo de seguridad criptográfica diseñado para garantizar la
privacidad y la integridad de los datos en comunicaciones sobre redes, como
Internet \cite{tls}. Opera sobre la capa de transporte y permite autenticación,
cifrado de datos y protección contra manipulación.

TLS se utiliza para garantizar la confidencialidad de los protocolos de
aplicación (MQTT, HTTP y WebSocket) \cite{awsiot_tls}.

\subsection{HTTP}

HTTP es un protocolo a nivel de aplicación que opera sobre TCP/IP y está
diseñado para sistemas de información distribuidos, colaborativos e hipermedia.
Está basado en el modelo cliente-servidor, diseñado para la transferencia de
recursos web \cite{rfc2616}.

Este protocolo es de naturaleza asíncrona: cuando un cliente envía una
petición, no necesita mantener una conexión activa mientras espera la
respuesta, lo que optimiza significativamente el uso de recursos en la red.
Este diseño sin estado hace que cada interacción sea independiente.

\subsection{WebSocket}

WebSocket es un protocolo de comunicación bidireccional y full-duplex que
establece una conexión persistente entre un cliente y un servidor sobre una
única conexión TCP \cite{RFC6455}. A diferencia del modelo de HTTP, WebSocket
permite el intercambio de datos en tiempo real sin necesidad de reabrir la
conexión en cada mensaje. Esto lo hace ideal para aplicaciones que requieren
baja latencia y actualizaciones instantáneas, como chats, juegos en línea y
monitoreo en tiempo real.

%----------------------------------------------------------------------------------------
%	SECTION 2 - Componentes de hardware
%----------------------------------------------------------------------------------------

\section{Componentes de hardware}\label{sec:hardware}

En esta sección se describen los diferentes elementos de hardware utilizados en
el desarrollo del trabajo. % La tabla \ref{tab:hardware} presenta los principales
% componentes utilizados y su función.

% \begin{table}[h]
% 	\centering
% 	\caption[Principales componentes hardware utilizados]{Principales componentes hardware utilizados}
% 	\begin{tabular}{p{3.2cm}p{9.6cm}}
% 		\toprule
% 		\textbf{Protocolo}                & \textbf{Función}                                                                                        \\
% 		\midrule
% 		\multirow{1}{*}{Módulo ESP32}     & Microcontrolador con Wi-Fi y Bluetooth integrado para conectividad y procesamiento central del sistema. \\
% 		\multirow{1}{*}{Sensor BME280}    & Mide temperatura ambiental, humedad relativa y presión atmosférica con alta precisión.                  \\
% 		\multirow{1}{*}{Sensor BH1750}    & Sensor de intensidad luminosa que mide la iluminación ambiental en lux.                                 \\
% 		\multirow{1}{*}{Sensor MHZ19C}    & Detector de dióxido de carbono $CO_2$ por infrarrojo.                                                   \\
% 		\multirow{1}{*}{Sensor de pH}     & Mide la acidez o alcalinidad de la solución nutritiva mediante electrodo.                               \\
% 		\multirow{1}{*}{Sensor de CE}     & Determina la conductividad eléctrica de la solución para estimar cantidad de nutrientes.                \\
% 		\multirow{1}{*}{Sensor de TDS}    & Mide sólidos disueltos totales en líquidos, relacionado con la concentración de nutrientes.             \\
% 		\multirow{1}{*}{Sensor DS18B20}   & Sensor digital de temperatura sumergible para líquidos.                                                 \\
% 		\multirow{1}{*}{Sensor HC-SR04}   & Mide distancias por ultrasonido (nivel de agua en depósitos).                                           \\
% 		\multirow{1}{*}{Sensor PZEM-004T} & Módulo de medición de parámetros eléctricos (tensión, corriente, potencia).                             \\
% 		\multirow{1}{*}{Relay 2 canales } & Actuador eléctrico para control ON/OFF de dispositivos (bombas, luces, etc.).                           \\
% 		\bottomrule
% 		\hline
% 	\end{tabular}
% 	\label{tab:hardware}
% \end{table}

\subsection{Microcontrolador}\label{sec:microcontrolador}

Para el desarrollo de los nodos se utilizó el microcontrolador ESP-WROOM-32, un
chip de bajo costo y bajo consumo de energía que integra Wi-Fi, Bluetooth y
Bluetooth LE en un solo paquete. El ESP-WROOM-32 es un microcontrolador de 32
bits con una arquitectura Xtensa LX6 de doble núcleo, que permite ejecutar dos
hilos de ejecución simultáneos. Además, cuenta con una amplia gama de
periféricos, como UART, SPI, I2C, ADC, DAC, PWM, entre otros, que lo hacen
ideal para aplicaciones de IoT \cite{EspressifESP32WROOM}.

En la siguiente figura \ref{fig:ESP32} puede observarse el módulo.

\begin{figure}[H]
	\centering
	\includegraphics[width=.30\textwidth]{./Images/3.png}
	\caption{Microcontrolador ESP-WROOM-32.}
	\label{fig:ESP32}
\end{figure}

\subsection{Sensor de temperatura ambiente, humedad relativa y presión atmosférica}

El sensor BME280 (figura \ref{fig:BME280}) es un sensor de temperatura, humedad
y presión atmosférica digital de alta precisión. Se comunica a través de la
interfaz I2C y SPI, y es capaz de medir la temperatura ambiente con una
precisión de ±1°C, la humedad relativa con una precisión de ±3\code{\%}, y la
presión atmosférica con una precisión de ±1 hPa\cite{BoschBME280}.

\begin{figure}[H]
	\centering
	\includegraphics[width=.15\textwidth]{./Images/4.png}
	\caption{Sensor BME280.}
	\label{fig:BME280}
\end{figure}

\subsection{Sensor de luz digital}\label{sec:BH1750}

El sensor BH1750 (figura \ref{fig:BH1750}) es un sensor de intensidad luminosa
digital que mide la iluminación ambiental en lux. Se comunica a través de la
interfaz I2C y es capaz de medir la intensidad luminosa en un rango de 1 a
65535 lux con una precisión de 1 lux \cite{ROHM_BH1750}.

\begin{figure}[H]
	\centering
	\includegraphics[width=.20\textwidth]{./Images/5.png}
	\caption{Sensor BH1750.}
	\label{fig:BH1750}
\end{figure}

\subsection{Sensor infrarrojo de $CO_2$}

El sensor MH-Z19C (figura \ref{fig:MHZ19C}) es un detector de dióxido de
carbono ($CO_2$) por NDIR (\textit{Non Dispersive Infrared Detector}). Se
comunica a través de la interfaz UART y es capaz de medir la concentración de
$CO_2$ en un rango de 0 a 5000 ppm con una precisión de 50 ppm
\cite{WINSEN_MHZ19C}.

\begin{figure}[H]
	\centering
	\includegraphics[width=.15\textwidth]{./Images/6.png}
	\caption{Sensor MHZ19C.}
	\label{fig:MHZ19C}
\end{figure}

\subsection{Sensor de detección de pH}

El sensor PH-4502C (figura \ref{fig:PH4502C}) mide la acidez o alcalinidad del
líquido mediante un electrodo de vidrio. Se comunica a través de la interfaz
analógica y es capaz de medir el pH en un rango de 0 a 14 \cite{PH-4502C}.

\begin{figure}[H]
	\centering
	\includegraphics[width=.20\textwidth]{./Images/7.png}
	\caption{Sensor PH-4502C.}
	\label{fig:PH4502C}
\end{figure}

\subsection{Sensor de conductividad eléctrica}

El sensor de CE (figura \ref{fig:CE}) mide la conductividad eléctrica del
líquido para estimar la cantidad de nutrientes disueltos en el agua. Se
comunica a través de la interfaz analógica y es capaz de medir la conductividad
en un rango de 0 a 20 mS/cm \cite{EC-Sensor}.

\begin{figure}[H]
	\centering
	\includegraphics[width=.20\textwidth]{./Images/8.png}
	\caption{Sensor CE.}
	\label{fig:CE}
\end{figure}

\subsection{Sensor de sólidos disueltos totales}

El sensor de TDS (figura \ref{fig:TDS}) mide la cantidad de sólidos disueltos
totales en el agua, relacionado con la concentración de nutrientes. Se comunica
a través de la interfaz analógica y es capaz de medir la concentración de TDS
en un rango de 0 a 1000 ppm \cite{TDS-Sensor}.

\begin{figure}[H]
	\centering
	\includegraphics[width=.20\textwidth]{./Images/9.png}
	\caption{Sensor TDS.}
	\label{fig:TDS}
\end{figure}

\subsection{Sensor de temperatura digital sumergible}

El sensor DS18B20 (figura \ref{fig:DS18B20}) es un sensor digital de
temperatura sumergible en líquidos. Se comunica a través de la interfaz 1-Wire
y es capaz de medir la temperatura en un rango de -55°C a 125°C con una
precisión de ±0.5°C \cite{DS18B20}.

\begin{figure}[H]
	\centering
	\includegraphics[width=.20\textwidth]{./Images/10.png}
	\caption{Sensor de temperatura DS18B20.}
	\label{fig:DS18B20}
\end{figure}

\subsection{Sensor ultrasónico}

El sensor HC-SR04 (figura \ref{fig:HC-SR04}) mide distancias por ultrasonido en
un rango de 2 cm a 400 cm con una precisión de 3 mm. Se comunica a través de la
interfaz GPIO \cite{HC-SR04}.

\begin{figure}[H]
	\centering
	\includegraphics[width=.20\textwidth]{./Images/11.png}
	\caption{Sensor HC-SR04.}
	\label{fig:HC-SR04}
\end{figure}

\subsection{Sensor de medición de consumo eléctrico}

El sensor PZEM-004T (figura \ref{fig:PZEM-004T}) es un módulo de medición de
parámetros eléctricos que mide la tensión, corriente, potencia activa y energía
consumida. Se comunica a través de la interfaz UART y es capaz de medir la
tensión en un rango de 80 a 260 V, la corriente en un rango de 0 a 100 A, y la
potencia en un rango de 0 a 22 kW \cite{PZEM-004T}.

\begin{figure}[H]
	\centering
	\includegraphics[width=.20\textwidth]{./Images/12.png}
	\caption{Sensor de medición de consumo eléctrico.}
	\label{fig:PZEM-004T}
\end{figure}

\subsection{Módulo Relay}

El módulo Relay (figura \ref{fig:Relay}) es un actuador eléctrico de dos
canales optocoplados que permite el control de encendido y apagado de
dispositivos eléctricos. Se comunica a través de la interfaz GPIO y es capaz de
controlar dispositivos de hasta 10 A y 250 VAC \cite{Relay}.

\begin{figure}[H]
	\centering
	\includegraphics[width=.20\textwidth]{./Images/13.png}
	\caption{Relay de 2 Canales 5v 10a}
	\label{fig:Relay}
\end{figure}

%----------------------------------------------------------------------------------------
%	SECTION 3 - Componentes de software
%----------------------------------------------------------------------------------------

\section{Desarrollo de firmware}

En esta sección se describe la herramienta de software utilizada para la
programación de los microcontroladores ESP32.

\subsection{MicroPython}

MicroPython es una implementación optimizada de Python 3 para
microcontroladores y sistemas embebidos. Está diseñado para ejecutarse en
dispositivos con recursos limitados, como el ESP32, y proporciona una forma
sencilla de programar microcontroladores utilizando un lenguaje de alto nivel
como Python \cite{MicroPython}.

Su facilidad de uso, la amplia disponibilidad de bibliotecas y la reducción del
tiempo de desarrollo lo convierten en una opción eficiente. Además, al ser un
lenguaje interpretado, posibilita la ejecución interactiva de pruebas y
depuración, facilitando la identificación y corrección de errores en el código.

\section{Desarrollo Backend y API}

En esta sección se presentan las herramientas de software utilizadas en el
desarrollo del backend y la API REST.

\subsection{FastAPI}

FastAPI es un framework moderno para la construcción de APIs REST rápidas y
escalables en Python. Está diseñado para ser fácil de usar, rápido de
desarrollar y altamente eficiente en términos de rendimiento. FastAPI utiliza
Python 3.6+ y aprovecha las características de tipado estático de Python para
proporcionar una API autodocumentada y con validación de tipos integrada
\cite{FastAPI}.

\subsection{MongoDB}

MongoDB es una base de datos NoSQL (\textit{Not Only SQL}) de código abierto y
orientada a documentos que proporciona una forma flexible y escalable de
almacenar y recuperar datos. Utiliza un modelo de datos basado en documentos
BSON (\textit{Binary JavaScript Object Notation}) que permite almacenar datos
de forma anidada y sin esquema fijo, lo que facilita la manipulación y consulta
de datos no estructurados \cite{MongoDB}.

\section{Desarrollo Frontend}

\subsection{React}

React es una biblioteca de JavaScript de código abierto para construir
interfaces de usuario interactivas y reutilizables. Desarrollada por Facebook,
React permite crear componentes de interfaz de usuario que se actualizan de
forma eficiente cuando cambian los datos, lo que facilita la creación de
aplicaciones web rápidas y dinámicas \cite{React}.

\section{Infraestructura y despliegue}

\subsection{Docker}

Docker es una plataforma de código abierto que permite a los desarrolladores y
a los equipos de operaciones construir, empaquetar y desplegar aplicaciones en
contenedores. Los contenedores son unidades de software ligeros y portátiles
que incluyen todo lo necesario para ejecutar una aplicación, incluidas las
bibliotecas, las dependencias y el código \cite{Docker}.

Docker facilita la creación de entornos de desarrollo y despliegue consistentes
y reproducibles, lo que garantiza que las aplicaciones se ejecuten de la misma
manera en cualquier entorno.

\subsection{AWS IoT Core}

AWS IoT Core es un servicio de AWS (\textit{Amazon Web Services}) que permite a
los dispositivos conectarse de forma segura a la nube y comunicarse entre sí a
través de protocolos de comunicación estándar como MQTT y HTTP. Proporciona una
infraestructura escalable y segura para la gestión de dispositivos, la
recopilación de datos y la integración con otros servicios de AWS
\cite{AWS_IoT}. Utiliza TLS para cifrar la comunicación entre los dispositivos
y la nube, garantizando la confidencialidad y la integridad de los datos.

\subsection{AWS EC2}

Amazon EC2 (\textit{Elastic Compute Cloud}) es un servicio de AWS que
proporciona capacidad informática escalable en la nube. Permite a los usuarios
lanzar instancias virtuales en la nube con diferentes configuraciones de CPU,
memoria, almacenamiento y red, lo que facilita la implementación de
aplicaciones escalables y de alta disponibilidad \cite{AWS_EC2}.

\section{Herramientas de desarrollo}

\subsection{Visual Studio Code}

Visual Studio Code, comúnmente abreviado como VS Code, es un entorno de
desarrollo integrado (IDE, \textit{integrated development environment}) de
código abierto, altamente extensible y multiplataforma compatible con Windows,
macOS y Linux \cite{VSCode}.

VS Code es un editor de código ligero y rápido con soporte para cientos de
lenguajes de programación y extensiones que permiten personalizar y mejorar la
funcionalidad del editor. Además, cuenta con herramientas de depuración
integradas, control de versiones y terminal integrada.

\subsection{Postman}

Postman es una plataforma de colaboración para el desarrollo de APIs que
permite a los desarrolladores diseñar, probar y documentar APIs de forma
rápida. Postman proporciona una interfaz gráfica intuitiva para enviar
solicitudes HTTP a un servidor y visualizar las respuestas, lo que facilita la
depuración y el desarrollo de APIs \cite{Postman}.

\subsection{Git}

Git es un sistema de control de versiones distribuido de código abierto
diseñado para gestionar proyectos de software de cualquier tamaño con rapidez y
eficiencia. Permite a los desarrolladores trabajar en paralelo en un mismo
proyecto, realizar seguimiento de los cambios, revertir a versiones anteriores
y colaborar en el desarrollo de software \cite{Git}.

\subsection{Github}

Github es una plataforma de alojamiento de repositorios Git que permite a los
desarrolladores colaborar en proyectos de software de forma distribuida.
Proporciona herramientas para gestionar el código fuente, realizar seguimiento
de los cambios, revisar el código, realizar integración continua y despliegue
automático \cite{Github}.

