\chapter{Conclusiones} % Main chapter title

\label{Chapter5}

Este capítulo presenta los resultados obtenidos sobre el trabajo realizado.
Además, se presentan mejoras como posibles trabajos futuros.

%----------------------------------------------------------------------------------------

%----------------------------------------------------------------------------------------
%	SECTION 1
%----------------------------------------------------------------------------------------

\section{Conclusiones generales}

Este trabajo logró con éxito el objetivo de diseñar e implementar un prototipo
funcional para el monitoreo y control remoto de condiciones climáticas en
invernaderos, mediante tecnologías IoT y una arquitectura centrada en el
usuario.

Se integraron sensores y actuadores capaces de relevar variables ambientales,
parámetros de la solución nutritiva, consumos de energía y nutrientes, así como
de controlar dispositivos a distancia, todo respaldado por una infraestructura
en la nube, segura y escalable.

El sistema desarrollado permite:
\begin{itemize}
      \item Registrar y consultar variables ambientales y parámetros del invernadero.
      \item Controlar actuadores desde una interfaz remota.
      \item Acceder a datos en tiempo real e históricos a través de una aplicación web
            responsiva.
\end{itemize}

Cada componente desarrollado, desde los sensores y actuadores hasta el servidor
y la aplicación web, cumplió con los requerimientos establecidos. La
incorporación del protocolo MQTT con cifrado TLS, junto con el uso de servicios
en la nube como AWS IoT Core y EC2, garantizó una comunicación segura,
eficiente y escalable.

La aplicación web, diseñada con un enfoque centrado en la experiencia del
usuario, resultó accesible, intuitiva y eficaz para la supervisión y el control
del sistema. El sistema permite asignar distintos roles a los usuarios: el
perfil administrador accede a todas las funciones, mientras que el usuario
estándar tiene acceso exclusivo a la visualización del monitoreo en tiempo real
y a los reportes. Esta diferenciación contribuye a una interacción segura y
adaptada a distintos perfiles.

La interfaz gráfica, desarrollada con tecnologías modernas como React y
Bootstrap, facilitó la interacción con el sistema, y su funcionalidad favoreció
la toma de decisiones informadas y el aprovechamiento de los datos tanto en
entornos académicos como productivos.

Las pruebas realizadas en un entorno controlado confirmaron la estabilidad,
eficacia y adaptabilidad del sistema, y evidenciaron su potencial para mejorar
la eficiencia operativa, la sostenibilidad y la autonomía en la gestión de
cultivos bajo condiciones controladas.

El desarrollo del prototipo permitió aplicar conocimientos en sistemas
embebidos, comunicación segura, desarrollo web, gestión de bases de datos y
despliegue de servicios en la nube. Esta experiencia constituye una base sólida
para futuras investigaciones y desarrollos tecnológicos en el ámbito de la
agricultura de precisión y el monitoreo ambiental.

%----------------------------------------------------------------------------------------
%	SECTION 2
%----------------------------------------------------------------------------------------
\section{Próximos pasos}

A continuación, se proponen líneas de trabajo que permitirán continuar con el
desarrollo y perfeccionar el sistema prototipado:

\begin{itemize}
      \item Optimización del firmware: reescribir el firmware en lenguaje C con el entorno
            de Espressif, para mejorar el rendimiento, la eficiencia y la estabilidad del
            sistema.
      \item Migración a LoRaWAN: Reemplazar la conexión Wi-Fi por una red LoRaWAN para
            ampliar la cobertura en entornos remotos, reducir el consumo energético y
            escalar el número de nodos conectados.

      \item Actualizaciones OTA: incorporar un sistema de actualizaciones inalámbricas que
            permita aplicar mejoras sin intervención física.

      \item Automatización e inteligencia artificial: desarrollar un sistema de control
            automático basado en reglas y aplicar técnicas de inteligencia artificial para
            optimizar el uso de recursos y mejorar la toma de decisiones.

      \item Control local: implementar un sistema autónomo que funcione sin conexión a
            Internet, con capacidad de decisión in situ.

      \item Validación en entornos reales: ejecutar pruebas piloto en invernaderos
            productivos para evaluar el desempeño del sistema en condiciones operativas
            reales.

      \item Gestión de roles y permisos: implementar la creación de roles y la asignación
            de permisos específicos a cada usuario.

      \item Aplicación móvil: desarrollar una app para dispositivos móviles que complemente
            la aplicación web y optimice la experiencia del usuario.

      \item Expansión funcional: ampliar el sistema con sensores adicionales y evaluar su
            aplicabilidad en otros contextos, como la gestión de recursos hídricos u otros
            cultivos.
\end{itemize}
