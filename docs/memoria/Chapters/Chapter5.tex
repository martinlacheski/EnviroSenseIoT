\chapter{Conclusiones} % Main chapter title

\label{Chapter5}

Este capítulo presenta los resultados obtenidos sobre el trabajo realizado.
Además, se presentan mejoras como posibles trabajos futuros.

%----------------------------------------------------------------------------------------

%----------------------------------------------------------------------------------------
%	SECTION 1
%----------------------------------------------------------------------------------------

\section{Conclusiones generales}

Este trabajo alcanzó con éxito el objetivo de diseñar e implementar un
prototipo funcional para el monitoreo y control remoto de condiciones
climáticas en invernaderos, mediante tecnologías IoT y una arquitectura
centrada en el usuario. Se integraron sensores y actuadores capaces de
gestionar variables ambientales, parámetros de la solución nutritiva, consumos
de energía y nutrientes, y el control de dispositivos a distancia, todo
respaldado por una infraestructura en la nube, segura y escalable.

El sistema desarrollado permite:
\begin{itemize}
    \item Registrar y consultar variables ambientales y parámetros del invernadero.
    \item Controlar actuadores desde una interfaz remota.
    \item Acceder a datos en tiempo real e históricos a través de una aplicación web
          responsiva.
\end{itemize}

Cada componente desarrollado, desde los sensores y actuadores hasta el servidor
y la aplicación web, cumplió con los requerimientos técnicos establecidos. La
incorporación del protocolo MQTT con cifrado TLS, junto con el uso de servicios
en la nube como AWS IoT Core y EC2, garantizó una comunicación segura,
eficiente y escalable.

La aplicación web, diseñada con un enfoque centrado en la experiencia del
usuario, resultó accesible, intuitiva y eficaz para la supervisión y el control
del sistema. El sistema permite asignar distintos roles a los usuarios: el
perfil administrador accede a todas las funciones, mientras que el usuario
estándar tiene acceso exclusivo a la visualización del monitoreo en tiempo real
y a los reportes. Esta diferenciación contribuye a una interacción segura y
adaptada a distintos perfiles.

La interfaz gráfica, desarrollada con tecnologías modernas como React y
Bootstrap, facilitó la interacción con el sistema, y su funcionalidad favoreció
la toma de decisiones informadas y el aprovechamiento de los datos tanto en
entornos académicos como productivos.

Las pruebas realizadas en un entorno controlado confirmaron la estabilidad,
eficacia y adaptabilidad del sistema, y evidenciaron su potencial para mejorar
la eficiencia operativa, la sostenibilidad y la autonomía en la gestión de
cultivos bajo condiciones controladas.

El desarrollo del prototipo permitió aplicar conocimientos en sistemas
embebidos, comunicación segura, desarrollo web, gestión de bases de datos y
despliegue de servicios en la nube. Esta experiencia constituye una base sólida
para futuras investigaciones y desarrollos tecnológicos en el ámbito de la
agricultura de precisión y el monitoreo ambiental.

%----------------------------------------------------------------------------------------
%	SECTION 2
%----------------------------------------------------------------------------------------
% \section{Próximos pasos}

% A continuación se proponen líneas de trabajo que permitirían continuar el
% desarrollo y mejorar el sistema prototipado:

% \begin{itemize}
%     \item Optimización del firmware: Reescribir el firmware de los nodos en lenguaje C,
%           con el IDE de Espressif, con el objetivo de mejorar el rendimiento, la
%           eficiencia en el uso de recursos y la estabilidad del sistema.
%     \item Incorporacion de actualizaciones OTA: Implementar un sistema de actualizaciones
%           inalámbricas (Over-The-Air) para el firmware de los nodos, para realizar
%           mejoras y correcciones sin necesidad de intervención física en el dispositivo.
%     \item Diseñar producto: Desarrollar un diseño industrial para el prototipo, que
%           contemple aspectos estéticos y funcionales, así como la integración de los
%           componentes electrónicos en una carcasa adecuada para su uso en campo.
%     \item Automatización e inteligencia: Incorporar un sistema de control automático
%           basado en reglas o algoritmos de aprendizaje automático, con el fin de
%           optimizar el uso de recursos. Además, aplicar técnicas de análisis de datos e
%           inteligencia artificial para extraer patrones relevantes y mejorar la toma de
%           decisiones.
%     \item Incorporación de control local: Implementar un sistema de control local que
%           permita operar el sistema de forma autónoma, sin necesidad de conexión a
%           Internet. Evaluar la integración de inteligencia artificial para la toma de
%           decisiones.
%     \item Integración de sensores adicionales: Ampliar la gama de sensores para integrar
%           captura de imagenes, humedad del suelo, entre otros, con el fin de obtener un
%           monitoreo más completo y preciso de las condiciones del invernadero.
%     \item Alertas y monitoreo proactivo: Integrar un sistema de notificaciones que alerte
%           a los usuarios ante eventos críticos o condiciones anómalas en el invernadero.
%     \item Interfaz y accesibilidad: Desarrollar una aplicación móvil que complemente la
%           plataforma web, para brindar mayor accesibilidad desde dispositivos móviles
%           para el monitoreo y control en campo.
%     \item Validación en entornos reales: Implementar pruebas piloto en invernaderos
%           productivos, con el propósito de validar el sistema bajo condiciones reales y
%           diversas de operación.
%     \item Gestión energética y sustentabilidad: Analizar alternativas para incorporar un
%           sistema de gestión de energía que permita reducir el consumo eléctrico y
%           mejorar la sostenibilidad del sistema, con vistas a disminuir su huella de
%           carbono.
% \end{itemize}

\section{Próximos pasos}

A continuación, se proponen líneas de trabajo que permitirán continuar con el
desarrollo y perfeccionar el sistema prototipado:

\begin{itemize}
    \item Optimización del firmware: reescribir el firmware en lenguaje C con el entorno
          de Espressif, para mejorar el rendimiento, la eficiencia y la estabilidad del
          sistema.

    \item Actualizaciones OTA: incorporar un sistema de actualizaciones inalámbricas que
          permita aplicar mejoras sin intervención física.

    \item Automatización e inteligencia artificial: desarrollar un sistema de control
          automático basado en reglas y aplicar técnicas de inteligencia artificial para
          optimizar el uso de recursos y mejorar la toma de decisiones.

    \item Control local: implementar un sistema autónomo que funcione sin conexión a
          Internet, con capacidad de decisión in situ.

    \item Validación en entornos reales: ejecutar pruebas piloto en invernaderos
          productivos para evaluar el desempeño del sistema en condiciones operativas
          reales.

    \item Interfaz móvil: desarrollar una aplicación para dispositivos móviles que
          complemente la plataforma web y facilite el acceso en campo.

    \item Expansión funcional: ampliar el sistema con sensores adicionales y evaluar su
          aplicabilidad en otros contextos, como la gestión de recursos hídricos u otros
          cultivos.
\end{itemize}

% \begin{itemize}
%     \item Optimización del firmware: Reescribir el firmware de los nodos en lenguaje C,
%           con el entorno de desarrollo de Espressif, para mejorar el rendimiento, la
%           eficiencia en el uso de recursos y la estabilidad general del sistema.
%     \item Incorporación de actualizaciones: Implementar un sistema de actualizaciones
%           inalámbricas OTA (del inglés, \textit{Over The Air}) que permita aplicar
%           mejoras y correcciones al firmware sin requerir intervención física sobre los
%           dispositivos.
%     \item Diseño del producto: Desarrollar una propuesta de diseño industrial para el
%           prototipo, para considerar aspectos funcionales como estéticos, e integrar los
%           componentes electrónicos en una carcasa adecuada para condiciones de uso en
%           campo.
%     \item Automatización: Incorporar un sistema de control automático basado en reglas
%           predefinidas, con el propósito de optimizar el uso de recursos.
%     \item Inteligencia artificial: aplicar técnicas de análisis de datos e inteligencia
%           artificial para identificar patrones y mejorar la toma de decisiones.
%     \item Incorporación de control local: Desarrollar un mecanismo de control autónomo
%           que garantice el funcionamiento del sistema sin dependencia de la conexión a
%           Internet, con la posibilidad de integrar inteligencia artificial para la toma
%           de decisiones in-situ.
%     \item Integración de sensores adicionales: Ampliar el conjunto de sensores
%           disponibles mediante la incorporación de cámaras, sensores de humedad del suelo
%           y otros dispositivos relevantes, con el fin de lograr un monitoreo más completo
%           y preciso del entorno del invernadero.
%     \item Alertas y monitoreo proactivo: Integrar un sistema de notificaciones que
%           informe a los usuarios sobre eventos críticos o condiciones anómalas detectadas
%           en el sistema.
%     \item Interfaz y accesibilidad: Desarrollar una aplicación móvil que complemente la
%           plataforma web, para facilitar el acceso y control del sistema desde
%           dispositivos móviles en entornos de producción.
%     \item Validación en entornos reales: Realizar pruebas piloto en invernaderos
%           productivos que permitan evaluar el desempeño y la efectividad del sistema en
%           condiciones operativas reales y variadas.
%     \item Gestión energética y sostenibilidad: Evaluar alternativas para incorporar un
%           sistema de gestión de energía que contribuya a reducir el consumo eléctrico y
%           mejorar la sostenibilidad del sistema, con el objetivo de disminuir su huella
%           de carbono.
%     \item Evaluación de nuevas áreas de aplicación: Analizar la viabilidad de adaptar el
%           sistema a otros contextos, como la gestión de recursos hídricos o diferentes
%           productos agrícolas, para ampliar su aplicabilidad y beneficios.
% \end{itemize}
